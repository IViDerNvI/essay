\section{概括归纳}

\subsection{概括归纳的基本原则}

\paragraph{基本原则} 优先使用文章中的标准化语言进行概括归纳。如果文章中没有标准化语言,则需要转化为标准化语言进行概括归纳语言。一般而言文段结构为总分总结构,在分部分中以具体案例体现,如果存在总结句可以提炼使用,如果没有总结句则需要概括案例特征进行总结。题目中强调了归纳时要求尽可能写小标题。

\subsection{概括标准化} 见附录\ref{appendix::1}。

\subsection{概括归纳的格式}

\paragraph{格式要求} 概括归纳的格式要求如下:大标号为单个阿拉伯数字,二级标题为括号括起的阿拉伯数字,均占用一个格子;大标号做分类,二级标题做具体内容概括。概括归纳的内容需要使用标准化语言进行概括,下举1-2个具体案例进行说明。如果字数限制在200字左右可以不写小标题。

\subsection{概括归纳的题型和解题思路}

\subsubsection{概括变化}

\paragraph{题型特点} 概括变化的题目通常会给出一段文字,要求考生对其中的变化进行概括。变化可以是时间上的、空间上的或事物性质上的变化。

\paragraph{解题思路} 针对精神文化,生态环境,经济发展,自然环境,民生福祉,治理模式等方面的变化进行概括。

